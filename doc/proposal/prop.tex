\documentclass[aip,floatfix,twocolumn,nofootinbib]{revtex4-1}
\usepackage{amsmath}
\usepackage{graphicx}
\usepackage{epsfig}
\begin{document}
\title{Texas CPF cosmic ray muon telescope toy simulation}
\author{Douglas Davis}
\affiliation{Department of Physics, The University of Texas at Austin}
\date{\today}
\begin{abstract}
The High Energy Physics group in the Center for Particles and Fields at UT Austin is currently constructing a cosmic ray muon tracking telescope. It is a 1024 channel detector utilizing the scintillation of extruded plastic and photomultiplier tubes to detect and track atmospheric muons. In order to develop and measure the accuracy of track reconstruction algorithms, a simulation to produce data in the form of real muon coincidence is required. Because it is a tracking detector, no real physical processes need to be simulated; only detector response needs to be simulated, a ``toy simulation''. The simulation will ``throw'' lines (a toy muon) through a 3D system, and intersect extrusions located at points in space which model the real detector. The extrusions which have the ``muon'' line intersect it will be a ``hit,'' because in the real detector that extrusion will have light scintillate, go to an optical fiber, and reach a photomultiplier tube, making that extrusion a hit. The simulation will be an object oriented software written in C++, relying heavily on trigonometry, where properties of the line will be defined or randomized to generate data based on detector response, which can then be given to a reconstruction software. An external High Energy Physics based software framework, \texttt{ROOT} (\texttt{http://root.cern.ch}) will be used for random number generation, visualization, and data storage.
\end{abstract}
\maketitle
\end{document}
